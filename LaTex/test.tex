\documentclass{beamer}
\usepackage{ctex}

\usetheme{Berlin}


\title{从「信息」的角度看动态规划}
\subtitle{洛谷网校五一课程 - 动态规划篇}

\author{Ruan Xingzhi}
\institute{洛谷网校}

\begin{document}

\begin{titlepage}
    
\end{titlepage}

\begin{frame}
    \frametitle{Update}

    \begin{itemize}
        \item 黑科技降噪,以后不会有噪音了(但只能在windows下用)
        \item 扯了一条长达10m的网线连我笔记本电脑,不会丢帧了
        \item 画质大提升
    \end{itemize}

\end{frame}

\begin{frame}
    \frametitle{Intro}

    Let's focus on the big picture.

    动态规划(DP)不是某种具体算法,而是一种\textbf{思想}。

    ~\\

    核心在于:把大问题转化为小问题,利用小问题的解推断出大问题的解。

    带着这种思想,我们来学习DP.

\end{frame}

\begin{frame}
    \frametitle{信息}

    本篇教程与以往的DP教程区别于,我们这一次会不停地提到``信息''的概念。

    \begin{itemize}
        \item 想解决一个问题,需要掌握哪些信息?
        \item 如何用尽可能少的信息推断出问题的解?
        \item 如何把难以得到的信息,转化为容易得到的信息?
        \item 如何用有限的空间,存储大量的信息?
    \end{itemize}

    这种思路不止对DP有用,亦体现在各种其他算法上。


\end{frame}

\section{基本思想}

\subsection{几个小例子}

\begin{frame}
    \frametitle{上楼梯}

    今有$n$阶台阶。初始时站在0级,每次可以向上走1级或2级。问方案总数?
    \pause
    
    ~\\

    暴力模拟:枚举每一步采取什么方案。指数级复杂度。

    \pause

    为什么?

    \pause

    暴力模拟的时候,存储了“每一步采取了什么方案”,这是典型的没啥用的信息。

    \pause

    ~\\


    考虑更优的算法。如果以$f[x]$表示“从0级走到$x$级的方案数”,假设$f[1], f[2]... f[n-1]$全都已知,如何利用这些信息推出$f[n]$?

\end{frame}

\begin{frame}
    \frametitle{上楼梯}

    走到$f[n]$,要么是从$n-1$级走上来的,要么是从$n-2$级来的。依据加法原理

    \[f[n] = f[n-1] + f[n-2]\]

    这就是这个问题的递推式。

\end{frame}

\begin{frame}
    \frametitle{回顾}

    Q:我们想知道什么信息?

    A:从0级走到n级的方案数$f(n)$.

    ~\\

    Q:为了得到$f(n)$,我们需要哪些信息?
    
    A:知道$f(n)$和$f(n-1)$即可。

    ~\\

    Q:上面的递推式,能否求出所有的$f(x)$?

    A:可以。

\end{frame}

\begin{frame}
    \frametitle{求解过程}

    求出$f[3]$,需要$f[1], f[2]$的信息。以此类推,把依赖关系画成图:

    ~\\

    \includegraphics[width=\textwidth]{img/1-1.png}

\end{frame}

\begin{frame}
    \frametitle{硬币问题}

    今天你手上有无限的面值为1,5,11元的硬币。

    给定$n$,问:至少用多少枚硬币,可以恰好凑出$n$元?
    
    \begin{example}
        \begin{itemize}
            \item $n=15$时答案是3,构造方法为5+5+5
            \item $n=12$时答案是2,构造方法为11+1
        \end{itemize}

    \end{example}

\end{frame}

\begin{frame}
    \frametitle{硬币问题}

    用$f[x]$记录“凑出$x$元所需要的硬币数”。那么答案显然就是$f[n]$.

    如何求出f数组呢?$f[x]$等于什么?

    \begin{block}{提示}
        注意思考“$f[x]$从哪里来”。
    \end{block}

\end{frame}

\begin{frame}
    \frametitle{硬币问题}

    考虑一个具体的例子:凑出15元。

    为了凑出$15$元,我们最开始的时候,可以使用哪枚硬币?

    \begin{itemize}
        \item 假设用了1元硬币,那么接下来要凑出14元。共1+4=5枚
        \item 假设用了5元硬币,那么接下来要凑出10元。共1+2=3枚
        \item 假设用了11元硬币,那么接下来要凑出4元。共1+4=5枚
    \end{itemize}

    ~\\

    这三种方案,当然是选代价最低的,所以我们在这一次决策中,选择了5元硬币。

\end{frame}

\begin{frame}
    \frametitle{硬币问题}

    现在再来看,$f[x]$是“凑出x元需要的硬币数”,它等于什么?

    ~\\

    可供选择的决策方案如下:

    \begin{itemize}
        \item 先用一个1元硬币,代价$1+f[x-1]$
        \item 先用一个5元硬币,代价$1+f[x-5]$
        \item 先用一个11元硬币,代价$1+f[x-11]$
    \end{itemize}
    
    在上述方案里面,选择代价最低的就行!

\end{frame}

\begin{frame}
    \frametitle{硬币问题}

    所以有

    \[f[x] = \min \begin{cases}
        1+f[x-1] \\
        1+f[x-5] \\
        1+f[x-11] \\
    \end{cases} \]

\end{frame}

\begin{frame}
    \frametitle{初步总结:状态}

    我们用\textbf{“大事化小,小事化了”}的思想,解决了上楼梯问题和硬币问题。
    大事能转化成小事,是因为大事和小事都有一样的\textbf{形式}:

    \begin{block}{上楼梯问题}
        大问题:爬上$n$级有多少种方案

        小问题:爬上$n-1$级有多少种方案、爬上$n-2$级有多少种方案

        它们都是\textbf{“爬上$\times\times$级有多少种方案”}这一类问题。
    \end{block}

    \begin{block}{硬币问题}
        大问题:凑出$n$元钱的最少硬币数

        小问题:凑出$n-1,n-5,n-11$元的最少硬币数

        它们都是\textbf{“凑出$\times\times$元钱所需最少硬币数”}这一类问题。
    \end{block}

\end{frame}

\begin{frame}
    \frametitle{初步总结:状态}

    可见,只有大问题和小问题拥有\textbf{相同的形式},才能考虑大事化小。

    如果满足这个要求,那么我们遇到的每个问题,都可以很简洁地表达。我们把可能遇到的每种\textbf{“局面”}称为状态。

    \begin{example}
        硬币问题中,要表达“我们需要凑出$n$元钱”这个局面,可以设计状态:“$f[x]$表示凑出$x$元用的最少硬币数”。

        上楼梯问题中,设计状态:“$f[x]$表示走上$x$级的方案数”。
    \end{example}

    设计完状态之后,只要能\textbf{利用小状态的解求出大状态的解},就可以动手把题目做出来!

\end{frame}

\begin{frame}
    \frametitle{狗屁不通生成器问题}

    今有某人网上提交作业,打算随便糊弄。

    本来文本框里面是没有字的。他每次干以下两件事之一:

    \begin{itemize}
        \item 打一个字上去,文本长度加1。
        \item 把已有的所有的字复制一遍,文本长度翻倍。
    \end{itemize}

    他想打出恰好$n$个字,那他至少需要操作多少次?

\end{frame}

\begin{frame}
    \frametitle{设计状态}

    想打出10个字,最简方案是:

    \texttt{0 -> 1 -> 2 -> 4 -> 5 -> 10}

    ~\\

    设计状态:

    我们记``打出$n$个字所需要的最少操作次数''为$f(n)$.

    为了求出$f(n)$,需要什么信息?

\end{frame}

\begin{frame}
    \frametitle{求解}

    不难注意到

    $$f(n) = \min \begin{cases}
        f(n - 1), \\
        f(n/2)\quad \text{if} ~ 2\mid n
    \end{cases}$$

    而又有基础$f(0)=0$,故每一个$f(x)$都可求。

    代码怎么写?

\end{frame}

\subsection{LIS问题}

\begin{frame}
    \frametitle{LIS问题}

    数组的”最长上升子序列“是指:最长的那一个单调上升的子序列。

    例如:数组a:[1,3,4,2,7,6,8,5]的最长上升子序列是1,3,4,7,8.

    如何求数组的最长上升子序列的\textbf{长度}?

\end{frame}

\begin{frame}
    \frametitle{LIS问题}

    想用大事化小来做这道题,必须先设计状态。

    如何设计状态,来完整地描述当前遇到的局面?
    \pause

    ~\\

    \begin{block}{设计状态}
        以$f[x]$表示“以$a[x]$结尾的上升子序列,最长有多长”!

        那么,答案就是$f[1],f[2]...f[n]$里面的最大值。

        ~\\

        问题来了,如何求出f数组?提示:思考$f[x]$\textbf{从哪里来}。

    \end{block}

\end{frame}

\begin{frame}
    \frametitle{求出f数组}
    
    $f[x]$表达的是“以$a[x]$结尾的最长的上升子序列长度”。这个最长的子序列,一定是把$a[x]$接在某个上升子序列尾部形成的!

    \begin{example}
        数组a:[1,3,4,7,2,6,8,\textbf{5}]

        考虑f[8],它的来源有:
        \begin{itemize}
            \item 自己一个元素作为一个序列。长度为1.
            \item 接在a[1]后面。长度为f[1]+1=2
            \item 接在a[2]后面。长度为f[2]+1=3
            \item 接在a[3]后面。长度为f[3]+1=4
            \item 接在a[5]后面。长度为f[5]+1=3
        \end{itemize}
    \end{example}

\end{frame}

\begin{frame}
    \frametitle{求出f数组}

    此时,稍有常识的人都会看出,要得到$f[x]$,只需要看$a[x]$能接在哪些数的后面。
    
    也就是:

    \[f[x] = \max_{p<x , a[p]<a[x]} \{f[p]+1\}\]

    其中$p<x,a[p]<a[x]$的含义是:枚举在$x$前面的,$a[p]$又比$a[x]$小的那些$p$. 因为$a[x]$可以接到这些数的后面,形成一个更长的上升子序列。

\end{frame}

\begin{frame}
    \frametitle{初步总结:转移}

    在前面三个例题中,我们都是先设计好状态,然后给出了一套用小状态推出大状态解的方法。

    从一个状态的解,得知另一个状态的解,我们称之为\textbf{“状态转移”}。这个转移式子称为“状态转移方程”。

    \begin{example}
        硬币问题中,状态转移方程是:
        \[ f[x] = 1+\min\{f[x-1], f[x-5], f[x-11]\} \]

        LIS问题中,状态转移方程是:
        \[f[x] = \max_{p<x , a[p]<a[x]} \{f[p]+1\}\]
    

    \end{example}

\end{frame}

\begin{frame}
    \frametitle{小结:状态和转移}

    总结刚刚学习的内容。如果我们想用大事化小的思想解决一个问题,我们需要:

    \begin{enumerate}
        \item 设计状态。把面临的每一个问题,用状态表达出来。
        \item 设计转移。写出状态转移方程,从而利用小问题的解推出大问题的解。
    \end{enumerate}

\end{frame}

\subsection{如何设计转移}

\begin{frame}
    \frametitle{设计转移}

    前三个问题中,我们设计转移的时候,考虑的都是“这个局面是从哪过来的”。

    这是一种常见的思路:当前状态的解未知。需要用已经解决的状态,来推出当前状态的解。
    \pause

    ~\\

    DP还有另一种设计转移的思路:当前状态的解已知。需要利用这个解,去更新它能走到的状态。
    \pause

    ~\\

    这两种思路,一种是考虑“我从哪里来”,一种是考虑“我到哪里去”。两种手段都是能解决问题的!

\end{frame}

\begin{frame}
    \frametitle{最长公共子序列问题}

    今有俩数组$A,B$,记$\text{lcs}(A,B)$为A,B最长的公共子序列的长度。

    \begin{example}
        A = [1,3,2,4,7,5,7]
       
        B = [1,2,0,3,5,7,9]

        ~\\
        最长的公共子序列是 [1,2,5,7]

        \text{lcs}(A,B) = 4
    \end{example}

\end{frame}

\begin{frame}
    \frametitle{设计状态}

    记$f(x, y)$表示A的前x个元素,与B的前y个元素的LCS.

    获取$f(n, m)$需要什么信息?

    \pause

    \begin{itemize}
        \item $f(n, m)$可以是$f(n-1, m)$
        \item $f(n, m)$可以是$f(n, m-1)$
        \item 如果$a[n] = b[n]$,则$f(n)$可以取$f(n-1, m-1)+1$
    \end{itemize}

    ref. https://blog.csdn.net/hrn1216/article/details/51534607

\end{frame}

\begin{frame}
    \frametitle{上楼梯问题再讨论}

    如何用“我到哪里去”的转移手段,解决上楼梯问题?

    \pause
    \[\begin{aligned}
        f[x] \rightarrow & f[x+1] \\
        \rightarrow & f[x+2]
        \end{aligned}\]
    
    代码实现不难。

\end{frame}


\begin{frame}
    \frametitle{硬币问题再讨论}

    如何用“我到哪里去”的转移手段,解决硬币问题?

    \pause
    \[\begin{aligned}
        f[x] \rightarrow & f[x+1] \\
        \rightarrow & f[x+5] \\
        \rightarrow & f[x+11]
        \end{aligned}\]
    

\end{frame}


\begin{frame}
    \frametitle{LIS问题再讨论}

    如何用“我到哪里去”的转移手段,解决LIS问题?

    \pause
    
    \[f[x] \rightarrow f[p]\]

    其中$p>x, a[p]>a[x]$.
    
\end{frame}

\begin{frame}
    \frametitle{小结:设计转移}

    设计转移有两种方法。

    \begin{itemize}
        \item pull型(我从哪里来):对于一个没有求出解的状态,利用能走到它的状态,来得出它的解。
        \item push型(我到哪里去):对于一个已经求好了解的状态,拿去更新它能走到的状态。
    \end{itemize}

\end{frame}

\begin{frame}
    \frametitle{DP三连}

    综上所述,如果您想用DP解决一个问题,要干的事情可以总结为DP三连:

    \begin{itemize}
        \item 我是谁?(如何设计状态)
        \item 我从哪里来?(pull型转移)
        \item 我到哪里去?(push型转移)
    \end{itemize}

    两种转移方式中,只需要选择一个来设计转移即可。

\end{frame}

\section{Basic DP}

\subsection{记忆化搜索}

\begin{frame}
    \frametitle{斐波那契数列}

    众所周知,斐波那契数列是

    \[
        \begin{aligned}
            F[1]&=1 \\
            F[2]&=1 \\
            F[n]&=F[n-2]+F[n-1] 
        \end{aligned}\]

    假设严格按照定义,写一个递归的代码,复杂度是什么情况?

\end{frame}

\begin{frame}
    \frametitle{斐波那契数列}

    朴素代码如下:

    \includegraphics[width=.9\textwidth]{img/2-1.png}

    \textbf{请估算时间复杂度}。

\end{frame}

\begin{frame}
    \frametitle{斐波那契数列}

    我们遇到的最大的麻烦,是很多fib值被重新计算了。

    ~\\

    假设现在在计算fib(7),明明fib(5)只需要计算一次就可以;
    
    但是fib(7)要调用fib(6)和fib(5),fib(6)要调用fib(5),所以fib(5)莫名其妙被调用了两次。
    
    ~\\

    \textbf{如何避免这种情况?}

\end{frame}

\begin{frame}
    \frametitle{记忆化}

    我们引入记忆化:

    \begin{block}{记忆化搜索}
        调用fun(x)时:
        \begin{itemize}
            \item 如果fun(x)没有被计算过,则计算fun(x),并存储到mem[x]
            \item 如果fun(x)被计算过,则直接返回mem[x]
        \end{itemize}
    \end{block}

    记忆化搜索的复杂度如何?

\end{frame}

\begin{frame}
    \frametitle{又谈硬币问题}

    如何用记忆化搜索写出硬币问题?

    采用哪种转移方式最方便?

\end{frame}

\begin{frame}
    \frametitle{小结:记忆化搜索}

    \begin{itemize}
        \item \textbf{按顺序递推}和\textbf{记忆化搜索},是DP的两种高效实现方式。
        \item 记忆化搜索一般配套“我从哪里来”的转移方式。
    \end{itemize}

    \begin{block}{记忆化搜索的优势}
        \begin{itemize}
            \item 如果转移顺序不太好确定,则记忆化搜索可以帮你省一堆事。
            \item 有时候,记忆化搜索更节省时间、空间。因为不可能达到的状态是不会被搜索到的。
        \end{itemize}
    \end{block}

\end{frame}

\begin{frame}
    \frametitle{Function}

    https://www.luogu.com.cn/problem/P1464
    \pause

    记忆化搜索模板题。建议做一做。

\end{frame}

\end{document}